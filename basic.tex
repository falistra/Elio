\documentclass{article}

%\usepackage[utf8]{inputenc} is no longer required (since 2018)

%Set the font (output) encoding
%--------------------------------------
\usepackage[T1]{fontenc} %Not needed by LuaLaTeX or XeLaTeX
%--------------------------------------

%Italian-specific commands
%--------------------------------------
\usepackage[italian]{babel}
%Hyphenation rules
%--------------------------------------
\usepackage{hyphenat}
\hyphenation{mate-mati-ca recu-perare}

\begin{document}
\tableofcontents

\vspace{2cm} %Add a 2cm space

\begin{abstract}
Questo è un breve riassunto dei contenuti del 
documento scritto in italiano.
\end{abstract}

\section{Sezione introduttiva}
Questa è la prima sezione, possiamo aggiungere 
alcuni elementi aggiuntivi e tutto 
digitato correttamente. Inoltre, se una parola 
è troppo lunga e deve essere troncato 
babel cercherà per troncare correttamente 
a seconda della lingua.

\section{Teoremi Sezione}
Questa sezione è quello di vedere cosa succede con i comandi 
testo definendo

\[ \lim x =  \sin{\theta} + \max \{3.52, 4.22\} \]
\end{document}