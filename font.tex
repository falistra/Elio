\documentclass{article}
\usepackage{xcolor}
\usepackage{noto}
\usepackage{hyperref}
\title{Using Google Noto fonts}
\author{Overleaf}
\date{April 2021}

\begin{document}

\maketitle

\section{Introduction}
This example project uses the \href{https://ctan.org/pkg/noto?lang=en}{\color{blue}\texttt{noto}} package to typeset your document using Google's Noto fonts\footnote{\url{https://www.google.com/get/noto/}}:
\begin{itemize}
\item \verb|\textbf{bold}| produces \textbf{bold}
\item \verb|\textit{italic}| produces \textit{italic}
\item \verb|\textbf{\textit{bold italic}}| produces \textbf{\textit{bold italic}}
\item \verb|\emph{emphasis}| produces \emph{emphasis}
\item \verb|\textbf{\emph{bold italic}}| produces \textbf{\emph{bold italic}}
\end{itemize}

\subsection{Monospaced fonts}
{\fontsize{50}{60}\selectfont Foo!}
You can use Noto's monospaced fonts for \texttt{regular} and \texttt{\textbf{bold}} monospaced text.

\subsection{Sans serif fonts}
Here is some \textsf{text is typeset in a sans serif font} together with \textbf{\textsf{text typeset in bold sans serif}}.

\section{Further reading}
Documentation for the \texttt{noto} package can be found in its \href{http://mirrors.ctan.org/fonts/noto/README}{\color{blue}\texttt{readme} file on CTAN}.

\end{document}